\documentclass[a4paper]{article}

% Be TeX att använda korrekt fontkodning.
\usepackage[T1]{fontenc}

% Lite mindre marginaler, tack.
\usepackage[margin=3cm]{geometry}

% Hjälpsam mellanrumsdefinition:
\usepackage{xspace}

% Vi vill kunna referera till stadgarna och använder xr för detta.
\usepackage{xr}
\externaldocument[S-]{Stadgar}

% Definera ett litet hjälpmakro för hänvisning till CPU:s namn.
\newcommand\CPU[1][]{CPU\ifx&#1&\else:#1\fi\xspace}

% Vi vill ha tomrad istället för indentation för att separera paragrafer.
\setlength{\parindent}{0em}
\setlength{\parskip}{1em}

% And.... begin!
\begin{document}

% Definera en egen liten titelsektion.
\begin{center}
	{\Huge Utkast till styrdokument för} \\[0.5em]
	{\Huge Chalmers Programmeraruppslutning} \\[1em]
	{Kopplat till "Utkast till stadgar för Chalmers Programmeraruppslutning", revision 1} \\[1em]
	{Revision 1} \\
	{\copyright Anton Mårtensson, 2012}
\end{center}

Detta dokument utgör \CPU{s} styrdokument och reglerar tillsammans med stadgarna \CPU{s} verksamhet. Stadgarna och styrdokumentet skall anses vara en enhet och inget av dessa dokument skall bedömas utan att ta hänsyn till det andra dokumentet.

\section{Riktlinjer för stadgarna}
Då stadgar skrivs eller ändras skall de involverade försöka hålla stadgarna
\begin{description}
\item[Lättläsliga] Stadgarna är inte lagtext och bör hålla sig till någorlunda normal svenska. Detta gäller även styrdokumentet. Däremot bör båda dokument luta åt en formell ton.
\item[Korta] Om långa paragrafer behövs bör man överväga hurvida de kanske passar bättre in i styrdokumentet.
\item[Fria] Stadgarna skall inte låsa \CPU{s} verksamhet, utan utgör en grundstomme för att föreningen skall fungera. Mer detaljerade regler och riktlinjer bör föras in i detta styrdokument där de enklare kan ändras vid behov.
\end{description}

\section{Kommentarer till stadgarna}
% Inkludera kommentarer från stadgedokumentet.
\def\commentheader#1{\textbf{#1}\\}
\documentclass[a4paper]{article}

% Be TeX att använda korrekt fontkodning.
\usepackage[T1]{fontenc}

% Lite mindre marginaler, tack.
\usepackage[margin=3cm]{geometry}

% Hjälpsam mellanrumsdefinition:
\usepackage{xspace}

% En hjälpdefinition för att slippa upprepningar av namnet. Namnet skrivs ut fullt första gången i en paragraf varefter förkortningen används.
\newif\iffullCPU
\newcommand\CPU[1][]{%
	\iffullCPU%
		\fullCPUfalse%
		Chalmers Programmeraruppslutning#1%
	\else%
		CPU\ifx&#1&\else:#1\fi%
	\fi%
	\xspace%
}

% \countparagraph påbörjar automatisk numrering av paragrafer i sektionen.
\makeatletter
\newcommand\countparagraphs{%
	\everypar{%
		\@nobreakfalse%
		\refstepcounter{paragraph}%
		\hspace{-4em}%
		\makebox[3em][r]{\small\textparagraph\,\arabic{paragraph}}%
		\hspace{1em}%
		\fullCPUtrue
	}%
}
\renewcommand\thesection{\textsection\,\arabic{section}}
\renewcommand\theparagraph{\thesection\,\textparagraph\,\arabic{paragraph}}
\@addtoreset{paragraph}{section}
\makeatother

% Definera om \section till att påbörja numrerade paragrafer.
\let\oldsection\section
\makeatletter
\renewcommand\section[1]{%
	\everypar{}%
	\oldsection{#1}\vspace{-1em}%
	\countparagraphs%
	\fullCPUtrue%
}
\makeatother

% Definera ett kommando \comment som skriver ut en kommentar till en separat fil som sedan kan inkluderas i ett annat dokument med kommentarer etc.
\newwrite\commentfile
\immediate\openout\commentfile=\jobname.comments
\newcommand\comment[2]{%
	\label{#1}%
	\immediate\write\commentfile{\detokenize{\par\commentheader{\ref{S-#1} #1} #2}}%
}

% Vi vill ha tomrad istället för indentation för att separera paragrafer.
\setlength{\parindent}{0em}
\setlength{\parskip}{1em}

\clubpenalty=10000
\widowpenalty=10000

% And.... begin!
\begin{document}

% Definera en egen liten titelsektion.
\begin{center}
	{\Huge Utkast till stadgar för} \\[0.5em]
	{\Huge Chalmers Programmerar\-uppslutning} \\[1em]
	{Revision 1} \\
	{\copyright Anton Mårtensson, 2012}
\end{center}

% De faktiskta stadgarna:

\section{Beskrivning av föreningen}

\emph{Namn:} Föreningens namn är ``\CPU''.

Föreningens förkortade namn är ``CPU''.

\emph{Syfte:} \CPU[s] huvudsakliga syfte är att medlemmarna skall ha roligt med programmering.

\emph{Ort:} \CPU existerar på, i och kring Chalmers Tekniska Högskola, Göteborg.

\emph{Föreningstyp:} \CPU är en ideell förening.

\section{Obundenhet}

\CPU är religiöst och politiskt obunden.

\CPU skall sträva mot att ej vara bunden till en specifik sektion på Chalmers.

\section{År}

Verksamhets- och räkenskapsår gäller tiden från 1 januari till 31 december.

\section{Verksamhet}

\emph{Underhållning:} \CPU skall aktivt arbeta för att dess medlemmar skall finna programmering roligt, samt försöka sprida denna glädje.

\emph{Utbildning:} \CPU skall arbeta för att stödja och förbättra programmeringsutbildningen på Chalmers sektioner. Detta skall till största möjliga mån ske i samarbete med rörda sektioner.

\emph{Forum:} \CPU skall driva ett internetforum för både medlemmar och ickemedlemmar.

\section{Medlemskap}

\emph{Potentiell medlem:} Medlem i \CPU kan den vara som vill verka för \CPU[s] målsättning och accepterar \CPU[s] stadgar samt (a)~är medlem i Chalmers studentkår, (b)~har varit detta eller (c)~är på annat sätt kopplad till Chalmers eller någon av Chalmers sektioner.

\comment{Potentiell medlem}{Punkt (c) är med avsikt vagt beskriven för att tillåta lite spelrum. Ett exempel är en student från annan ort som för tillfället studerar på någon av Chalmers sektioner.}

\emph{Aktiv medlem:} För att bli medlem i \CPU krävs att den potentiella medlemmen registrerat sig på \CPU[s] internetforum samt deltagit i en CPU-aktivitet. Den nya medlemmen är en aktiv medlem.

\emph{CPU-aktivitet:} En ``CPU-aktivitet'' är en aktivitet, sammankomst eller händelse som officiellt arrangerats av eller i sammarbete med \CPU.

\emph{Passiv medlem:} En aktiv medlem som ej deltagit i en CPU-aktivitet under föregående eller nuvarande år övergår till att vara en passiv medlem.

En passiv medlem som deltar i en CPU-aktivitet övergår omedelbart till att vara en aktiv medlem.

\emph{Utgång ur CPU:} En passiv medlem som varit sådan under ett helt år slutar vara en medlem i \CPU. En person som utgått på detta sätt har rätt att återigen bli en medlem på normalt sätt.

\emph{Uteslutning:} En medlem som (a)~missbrukar sin position inom \CPU, (b)~skadar \CPU eller (c)~skadar \CPU[s] övriga medlemmar, kan bli utesluten ur \CPU. Årsmötet bestämmer om någon skall bli utesluten.

\comment{Uteslutning}{Uteslutning är en mycket allvarlig handling och bör endast genomföras i allvarliga fall.}

\comment{Uteslutning (a)}{Med att ``missbruka sin position inom \CPU'' menas exempelvis att medlemmen i fråga har utnyttjat sitt medlemskap inom \CPU gentemot en ickemedlem eller att en medlem har utnyttjat eventuell post inom \CPU gentemot en annan medlem eller ickemedlem.}

\emph{Tillsvidareuteslutning:} Om det fortsatta medlemskapet av en medlem som riskerar uteslutning anses utgöra en risk kan styrelsen besluta att tillfälligt utesluta denne medlem, som längst fram till årsmöte.

\comment{Tillsvidareuteslutning}{Styrelsen väntas vara rättvis och på intet sätt hålla en medlem utesluten efter att denne inte längre bedöms utgöra en risk. Styrelsen väntas också vara opartisk och om den aktuella medlemmen är sittande i styrelsen faller det på övriga styrelsen att bedöma situationen som om denne var en vanlig medlem.}

En utesluten medlem har ej rätt att på nytt bli medlem. Efterföljande årsmöten har rätt att bestämma att en utesluten medlem skall återgå till att vara en vanlig ickemedlem, varvid denne återigen kan bli medlem på normalt sätt.

\section{Styrelsen}

Styrelsen ansvarar för den löpande verksamheten och för verkställandet av årsmötets beslut och består av ordförande samt tre till åtta ledamöter.

\emph{Val av styrelse:} Styrelsen väljs vid det ordinarie årsmötet.

\emph{Styrelsemöte:} Styrelsen ska ha möte minst två gånger per verksamhetsår. Styrelsen får ta beslut på styrelsemötet om kallelse gått ut till hela styrelsen inom rimlig tid och minst halva styrelsen är närvarande.

\comment{Styrelsemöte}{Föreslagsvis är ``rimlig tid'' åtminstone en dag eller två, men om så mycket tid ej behövs krävs inte detta av stadgarna.}

\section{Årsmöte}

Årsmötet är \CPU[s] högsta beslutande möte och skall hållas en gång under läsperiod 3 varje år per Chalmers kalender.

\emph{Kallelse:} För att beslut skall kunna tas på årsmötet måste en kallelse skickats ut minst 14 dagar innan mötet börjar. Kallelse skall skickas till alla medlemmar eller finnas så att alla kan ta del av den.

Tid och plats för årsmötet bestäms av styrelsen, som också ansvarar för att kallelsen för mötet skickas ut.

\emph{Möteshandlingar:} Alla möteshandlingar skall finnas tillgängliga för alla närvarande innan mötet börjar.

\emph{Rösträtt:} Alla närvarande medlemmar har rösträtt. Varje medlem har endast en röst och denna kan inte överlåtas till någon annan.

\emph{Beslutsrätt:} För att en omröstning skall vara giltig krävs att minst en person från styrelsen närvarar och vinnande förslag måste ha minst fem röster.

\emph{Omröstning:} Vid omröstning vinner det förslag med flest bifall. I vissa fall finns det dock i stadgorna ytterligare krav för att det vinnande förslaget skall verkställas.

\emph{Yttranderätt:} Vid årsmöte har alla närvarande medlemmar rätt att yttra sig samt lägga förslag.

\emph{Extra årsmöte:} Ett extra årsmöte sammankallas om (a)~styrelsen begär det eller (b)~minst en fjärdedel av de aktiva medlemmarna skriftligen begär det. Alla regler för vanligt årsmöte gäller även för extra årsmöten.

\comment{Extra årsmöte}{Om extra årsmöte sammankallas på begäran av \CPU[s] medlemmar måste årsmötet ta upp de frågor som förorsakat denna begäran.}

\section{Revision}

Vid det ordinarie årsmötet ska välja en revisor och en revisorsuppleant. Dessa får inte sitta i styrelsen.

Revisorn har under verksamhetsåret i uppgift att övervaka styrelsen och \CPU[s] verksamhet för att bedöma huruvida vid årsmöte tagna beslut har verkställts.

Vid ordinarie årsmötet skall revisorn för det föregående verksamhetsåret presentera sin revisionsberättelse.

\section{Stadgar och styrdokument}

Detta dokument utgör \CPU[s] stadgar och reglerar tillsammans med styrdokumentet \CPU[s] verksamhet. Stadgarna och styrdokumentet skall anses vara en enhet och inget av dessa dokument skall bedömas utan att ta hänsyn till det andra dokumentet.

\emph{Offentlighet:} Stadgar och styrdokument skall finnas officiellt tillgängliga på lämpligt sätt. Vid ändring av någondera skall den nya versionen finnas tillgänglig senast en vecka efter ändring.

\comment{Offentlighet}{Exempelvis via en länk till dessa två dokument på \CPU[s] hemsida.}

\emph{Stadgeändring:} Dessa stadgar kan ändras genom att ett förslag till ändring röstas igenom vid årsmöte. För att en ändring skall verkställas krävs bifall från minst två tredjedelar av rösterna.

För att omröstning om stadgeändring skall få ske måste kallelsen ha inkluderat denna punkt och den föreslagna ändringen måste ha funnits tillgänglig för alla medlemmar från dess att kallelse gått ut.

\emph{Styrdokumentet:} Styrdokumentet är ett supplement till stadgarna och reglerar informellt \CPU[s] löpande verksamhet samt innehåller kontext, förtydlingar och tolk\-ningar av stadgarna.

\emph{Ändring av styrdokument:} Styrelsen har rätt att på ett styrelsemöte ändra styrdokumentet och väntas använda denna rätt för att hålla styrdokumentet uppdaterat.

\comment{Ändring av styrdokument}{Speciellt så väntas styrelsen uppdatera styrdokumentet med beslut rörande tolkning av stadgar och tagna beslut som långvarigt påverkar \CPU. Dessa uppdateringar och ändringar bör innehålla både kontext, resonemang och beslut.}

Vid ändring av styrdokumentet skall ett meddelande om att detta har skett finnas tillgänglig för alla medlemmar. 

\section{Upplösning}

Upplösning av \CPU kan bara ske genom att beslut om detta tas på ett ordinarie årsmöte då denna punkt inkluderats i kallelsen. För upplösning krävs minst två tredjedels bifall vid omröstning.

Vid upplösning skall \CPU[s] tillgångar doneras till likasinnade fören\-ingar eller organisationer. Vilka föreningar eller organisationer detta är samt hur tillgångarna skall fördelas mellan dessa bestäms på mötet.

% And... we're done.
\end{document}

% And... we're done.
\end{document}