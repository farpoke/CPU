\documentclass[a4paper]{article}

% Be TeX att använda korrekt fontkodning.
\usepackage[T1]{fontenc}

% Lite mindre marginaler, tack.
\usepackage[margin=3cm]{geometry}

% Hjälpsam mellanrumsdefinition:
\usepackage{xspace}

% En hjälpdefinition för att slippa upprepningar av namnet. Namnet skrivs ut fullt första gången i en paragraf varefter förkortningen används.
\newif\iffullCPU
\def\CPU{%
	\iffullCPU%
		\fullCPUfalse%
		Chalmers Programmeraruppslutning\xspace%
	\else%
		CPU\xspace%
	\fi%
}

% \countparagraph påbörjar automatisk numrering av paragrafer i sektionen.
\makeatletter
\newcommand\countparagraphs{%
	\everypar{%
		\@nobreakfalse%
		\addtocounter{paragraph}{1}%
		\hspace{-4em}%
		\makebox[3em][r]{\small\theparagraph}%
		\hspace{1em}%
		\fullCPUtrue
	}%
}
\renewcommand\thesection{\textsection\,\arabic{section}}
\renewcommand\theparagraph{\textparagraph\,\arabic{paragraph}}
\@addtoreset{paragraph}{section}
\makeatother

% Definera om \section till att påbörja numrerade paragrafer och ändra mellanrummen mellan sektioner något.
\let\oldsection\section
\makeatletter
\renewcommand\section[1]{%
	\everypar{}%
	\oldsection{#1}\vspace{-1em}%
	\countparagraphs%
	\fullCPUtrue%
}
\makeatother

% Vi vill ha tomrad istället för indentation för att separera paragrafer.
\setlength{\parindent}{0em}
\setlength{\parskip}{1em}

% And.... begin!
\begin{document}

% Definera en egen minimalistisk titelsektion.
\begin{center}
	{\Huge Utkast till stadgar för} \\[0.5em]
	{\Huge Chalmers Programmeraruppslutning} \\[1em]
	{Revision 1, 2012-04-10} \\
	{\copyright Anton Mårtensson, 2012}
\end{center}

% De faktiskta stadgarna:

\section{Beskrivning av föreningen}

\emph{Namn:} Föreningens namn är ``\CPU''.

Föreningens förkortade namn är ``CPU''.

\emph{Syfte:} \CPU{s} huvudsakliga syfte är att medlemmarna skall ha roligt med programmering.

\emph{Ort:} \CPU existerar på, i och kring Chalmers Tekniska Högskola, Göteborg.

\emph{Föreningstyp:} \CPU är en ideell förening.

\section{Obundenhet}

\CPU är religiöst och politiskt obunden.

\CPU skall sträva mot att ej vara bunden till en specifik sektion på Chalmers.

\section{År}

Verksamhets- och räkenskapsår gäller tiden från 1 januari till 31 december.

\section{Verksamhet}

\emph{Underhållning:} \CPU skall aktivt arbeta för att dess medlemmar skall finna programmering roligt, samt försöka sprida denna glädje.

\emph{Utbildning:} \CPU skall arbeta för att stödja och förbättra programmeringsutbildningen på Chalmers sektioner. Detta skall till största möjliga mån ske i samarbete med rörda sektioner.

\section{Medlemskap}

\emph{Potentiell medlem:} Medlem i \CPU kan den vara som håller med om föreningens syfte samt (a)~är medlem i Chalmers studentkår, (b)~har varit detta eller (c)~är på annat sätt kopplad till Chalmers eller någon av Chalmers sektioner (exempelvis en student från annan ort som för tillfället studerar på någon av Chalmers sektioner).

\emph{Aktiv medlem:} För att bli medlem i \CPU krävs att den potentiella medlemmen registrerat sig på \CPU{s} internetforum samt deltagit i en CPU-aktivitet. Den nya medlemmen är en aktiv medlem.

\emph{Passiv medlem:} En aktiv medlem som ej deltagit i en CPU-aktivitet under föregående eller nuvarande år övergår till att vara en passiv medlem.

En passiv medlem som deltar i en CPU-aktivitet övergår omedelbart till att vara en aktiv medlem.

\emph{Utgång ur CPU:} En passiv medlem som varit sådan under ett helt år slutar vara en medlem i \CPU. En person som utgått på detta sätt har rätt att återigen bli en medlem på normalt sätt.

\emph{Uteslutning:} En medlem som (a)~missköter sig inom \CPU, (b)~skadar \CPU eller (c)~skadar \CPU{s} övriga medlemmar, kan bli utesluten ur \CPU. Årsmötet bestämmer om någon skall bli utesluten.

\emph{Tillsvidareuteslutning:} Om det fortsatta medlemskapet av en medlem som riskerar uteslutning aktivt skadar \CPU eller \CPU{s} övriga medlemmar kan styrelsen besluta att tillfälligt utesluta denne medlem, som längst fram till årsmöte.

En utesluten medlem har ej rätt att på nytt bli medlem. Årsmötet har rätt att bestämma att en utesluten medlem skall återgå till att vara en vanlig ickemedlem, varvid denne återigen kan bli medlem på normalt sätt.

Styrelsen är skyldig att inte hålla en medlem utesluten mer än nödvändigt.

\section{Styrelsen}

Styrelsen ansvarar för den löpande verksamheten och för verkställandet av årsmötets beslut och består av ordförande samt tre till åtta ledarmoter.

\emph{Val av styrelse:} Styrelsen väljs vid det ordinarie årsmötet.

\emph{Styrelsemöte:} Styrelsen ska ha möte minst två gånger per verksamhetsår. Styrelsen får ta beslut på styrelsemötet om kallelse gått ut till hela styrelsen inom rimlig tid och minst halva styrelsen är närvarande.

\section{Årsmöte}

Årsmötet är \CPU{s} högsta beslutande möte och skall hållas en gång under läsperiod 3 varje år per Chalmers kalender. Årsmötet är en CPU-aktivitet.

\emph{Kallelse:} För att beslut skall kunna tas på årsmötet måste en kallelse skickats ut minst 14 dagar innan mötet börjar. Kallelse skall skickas till alla medlemmar eller finnas så att alla kan ta del av den.

Tid och plats för årsmötet bestäms av styrelsen, som också ansvarar för att kallelsen för mötet skickas ut.

\emph{Möteshandlingar:} Alla möteshandlingar skall finnas tillgängliga för alla närvarande innan mötet börjar.

\emph{Rösträtt:} Alla aktiva medlemmar har rösträtt. Varje medlem har endast en röst och denna kan inte överlåtas till någon annan.

\emph{Beslutsrätt:} För att en omröstning skall vara giltig krävs att minst en person från styrelsen närvarar och vinnande förslag måste ha minst fem röster.

\emph{Omröstning:} Vid omröstning vinner det förslag med flest bifall. I vissa fall finns det dock i stadgarna ytterligare krav för att det vinnande förslaget skall verkställas.

\emph{Yttranderätt:} Vid årsmöte har alla aktiva medlemmar rätt att yttra sig samt lägga förslag.

\emph{Extra årsmöte:} Ett extra årsmöte sammankallas om (a)~styrelsen begär det eller (b)~minst en fjärdedel av de aktiva medlemmarna skriftligen begär det. Alla regler för vanligt årsmöte gäller även för extra årsmöten.

\section{Revision}

Årsmötet ska välja en revisor och en revisorsuppleant. Dessa får inte sitta i styrelsen.

\section{Stadgeändring}

Dessa stadgar kan ändras genom att ett förslag till ändring röstas igenom på ordinarie årsmöte. För att en ändring skall verkställas krävs bifall från minst två tredjedelar av rösterna.

För att omröstning om stadgeändring skall få ske måste kallelsen ha inkluderat denna punkt och den föreslagna ändringen måste ha funnits tillgänglig för alla medlemmar från dess att kallelse gått ut.

\section{Upplösning}

Upplösning av \CPU kan bara ske genom att beslut om detta tas på ett ordinarie årsmöte då denna punkt inkluderats i kallelsen. För upplösning krävs minst två tredjedels bifall vid omröstning.

% And... we're done.
\end{document}